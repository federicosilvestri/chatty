% !TeX spellcheck = it_IT
\documentclass[]{article}
\usepackage[]{algorithm2e}
\usepackage{float}
\usepackage{listings}
\usepackage{hyperref}

%opening
\title{SOL - Progetto Chatty}
\author{Federico Silvestri 559014}

\begin{document}
\maketitle


\begin{abstract}
Questo documento contiene la documentazione relativa al progetto Chatty del modulo Laboratorio del corso A di Sistemi Operativi anno 2017/2018.
\end{abstract}
\tableofcontents
\pagebreak

\section{Architecture}
L'architettura utilizzata per lo sviluppo del progetto \`e basata sul modello \textbf{producer/consumer},
con l'aggiunta di un ulteriore componente \textbf{controller} che ha la responsabilit\`a di pilotare quest'ultimi.
Sia producer che consumer sono gestiti unicamente dal controller che offre un meccanismo per lo scambio delle informazioni
e ne assicura il sincronismo. Di seguito verranno descritte le funzionalit\`a pincipali di ogni singolo componente.

\subparagraph[Una breve descrizione del modello]{Il modello}
Prima di descrivere l'architettura componente per componente \`e necessario illustrare il modello base che viene seguito.
Tecnicamente parlando esso prevede l'implementazione dei seguenti metodi:

\begin{figure}[H]
	\begin{table}[H]
		\begin{tabular}{ll}			
			\textbf{name} & \textbf{description} \\
			component\_init() & lettura configurazione, allocazione memoria \\
			component\_start() & creazione del thread/thread pool \\
			component\_stop() & invio del segnale di stop al(la) thread/thread pool \\
			component\_destroy() & iberazione memoria, chiusura file descriptors \\
		\end{tabular}
	\end{table}
\end{figure}

Questi quattro metodi garantiscono la robustezza e la sicurezza in quanto ognuno di essi racchiude
tutte le operazioni necessarie affinch\`e la memoria sia allocata e liberata, i file descriptor
aperti e chiusi, i thread avviati e stoppati.

\paragraph{Producer}
Il producer generalmente \`e il componente che si occupa di produrre il lavoro da far eseguire al
consumer: in questo caso la produzione di lavoro \`e intesa come la ricezione di eventi sulle socket
aperte dai client.

Per fare un esempio: quando un client invia un messaggio di REGISTER\_OP, quest'ultimo provoca
l'attivazione del Producer che dovr\`a occuparsi di gestire tale richiesta in modo pi\`u
veloce e \textit{fair} possibile, anche se non dovr\`a essere lui stesso a doverla
processare.

Sintetizzando, le responsabilit\`a principali del Producer sono:
\begin{itemize}
	\item Creazione e binding della socket di ascolto
	\item Accettazione delle nuove connessioni
	\item Migrazione della richiesta al producer
\end{itemize}

Una volta che la richiesta \`e stata presa in carico dal consumer,
il producer non ha nessun altra responsabilit\`a su di essa.

\paragraph{Consumer}
Le richieste che vengono messe all'interno della coda di attesa, vengono
ricevute in modo sequenziale da una delle istanze del componente consumer
(in quanto multithread), che \`e l'endpoint del flusso di trasferimento della richiesta.

Il consumer si occupa dunque di processarla ed eventualmente di:
\begin{itemize}
	\item rimanere in attesa di un altro evento
	\item chiudere la connessione
	\item rilasciare la connessione al producer
\end{itemize}

\paragraph{Controller}
Siccome \textbf{P e C} non sono in grado di essere
autonomi, c'\`e bisogno di un ulteriore componente che coordina
le operazioni principali, ovvero che esegue al momento giusto
e nell'ordine giusto i metodi che sono stati descritti
nel modello base. In particolare i punti pi\`u critici sono lo startup e lo 
shutdown, in quanto viene gestita la memoria e controllata
la correttezza della configurazione passata al sistema.
 
Ipotiziamo per esempio che non venisse eseguito il metodo \textit{init()}.
Questo comporterebbe che la memoria non verrebbe allocata 
e dunque durante il runtime si genererebbe un errore, molto probabilmente
di segmentazione.
Altra ipotesi: se non venisse eseguito il metodo \textit{destroy},
la memoria utilizzata dai componenti non verrebbe deallocata,
generando di conseguenza i temuti \textit{memory leaks}, lo stesso discorso
vale anche per i file descriptor.
\\
\\
Le fasi di inizializzazione e distruzione non hanno un ordine ben preciso,
in generale tutti i componenti possono essere inizializzati o distrutti
anche concorrentemente. Ovviamente a differenza di queste le fasi
di avvio e stop devono essere eseguite secondo l'ordine prestabilito
dalle necessit\`a dei componenti.
In questo contesto il producer viene avviato prima del consumer, perch\`e
la gestione delle socket delle connessioni \`e affidata completamente al
producer ed \`e di fondamentale importanza per il consumer.

\paragraph{Implementazione del producer}
Il producer viene lanciato come un thread separato dal processo principale,
in modo da poterlo rendere indipendente da esso in termini di segnali e di interazione con l'utente.
Durante la fase di inizializzazione viene creata e bindata la socket sulla quale
arriveranno le richieste da parte dei client, controllando che non sia gi\`a stata
utilizzata da un altro processo in esecuzione.
Infatti un requisito fondamentale per l'avvio del producer \`e proprio quello appena descritto: la mutua esclusivit\`a sulla socket di ascolto.
Una volta controllate le condizioni sufficienti all'avvio, con il  metodo \textit{producer\_init()}, viene lanciato il thread producer,
che si occuper\`a di gestire le connessioni con il seguente algoritmo:

\begin{algorithm}[H] 
	\While{server\_status == RUNNING}{
		socksMutex.lock();
		init\_fds()\;
		socksMutex.unlock()\;
		select(read\_fds, write\_fds)\;
		socksMutex.lock()\;
		
		\eIf{newConn()}{
			manageNewConn()\;
		}{
			manageCurrentConns()\;
		}
		socksMutex.unlock()\;
	}
	\caption{The producer algorithm}
\end{algorithm}

La variabile \textit{socksMutex} fa riferimento al meccanismo di sincronizzazione
che viene utilizzato per la mutua esclusivit\`a dell'accesso al vettore delle socket.
Il vettore delle socket, chiamato \textit{sockets} all'interno del codice, contiene
$n$ interi ognuno dei quali rappresenta il filedescriptor di una specifica connessione.
Tale vettore viene messo sotto controllo da un mutex perch\`e il consumer stesso lo utilizza
per processare la richiesta dell'utente.
Una volta che una connessione \`e stata presa in carico da un consumer in esecuzione,
gli viene data la responsabilit\`a di tutti gli eventi relativi a quella socket, fino a
che non viene rilasciata dal consumer stesso.
Questo meccanismo viene implementato attraverso il vettore $socket_blocks$, che contiene
$n$ booleani, dove $n$ \`e il numero massimo di connessioni.

\paragraph{Implementazione del consumer}
A differenza del producer, il consumer invece pu\`o essere lanciato in modalit\`a multithread.
Infatti all'interno del file di configurazione il parametro \textit{ThreadsInPool} specifica
quanti thread dovranno essere creati per gestire tutte le richieste.
L'obiettivo del consumer \`e quello di occuparsi dell'evento che \`e ricevuto nella socket.
Per evento si intende:
\begin{itemize}
	\item la scrittura di un messaggio dal client
	\item la disponibilit\`a di spazio sul buffer
\end{itemize}
Nel caso in cui si parli di scrittura da parte del client, il consumer si occuper\`a di leggere
il messaggio ed eseguire tutte le procedure per completare la richiesta.
Nel in cui si parli di lettura il consumer invierà i messaggi disponibili al client.
Tutte queste funzionalit\`a sono state implementate nelle libreria nominata \textit{worker}.

\section{Librerie esterne}
Per lo sviluppo di questo progetto sono state utilizzate delle librerie esterne, non tanto per semplificare, ma per rendere
pi\`u efficiente e stabile possibile l'applicazione.

\subsection{RabbitMQ}
La gestione delle code \`e  stato uno degli argomenti informatici pi\`u richiesti e pi\`u studiati nel giro degli ultimi anni.
Per questo motivo sono stati sviluppati diversi software che forniscono un servizio di \textit{Message brokering},
ovvero di scambio di messaggi tra componenti di software diversi.
La scelta che \`e stata fatta per questo progetto \`e stata quella di usare \textit{RabbitMQ}, in quanto \`e uno dei 
pi\`u semplici e pi\`u user friendly con il linguaggio C.
RabbitMQ \`e dunque una dipendenza del progetto che viene installata automaticamente durante la fase di build di
chatty, e viene automaticamente disinstallata una volta lanciato lo script di pulizia.

\paragraph{Configurazione}
La configurazione di RabbitMQ avviene durante le fasi di inizializzazione del producer e del consumer.
In particolare vengono fatte due configurazioni diverse, ognuna delle quali \`e caratterizzata da una
connessione TCP dedicata verso il server RabbitMQ.

In generale quello che viene fatto per configurare l'applicazione sono le seguenti azioni:
\begin{itemize}
	\item creazione dell'exchage di tipo fanout
	\item creazione della coda per l'exchange
	\item binding della coda all'exchange
\end{itemize}
All'interno del file di configurazione possono essere configurati ulteriori parametri, che sono indicati
nella sezione configurazione.

\paragraph{Utilizzo}
RabbitMQ viene utilizzanto fondamentalmente nel progetto, ovvero esso \`e il mezzo di comunicazione tra
producer e consumer.
Quando il producer identifica l'arrivo di un evento all'interno della socket, compone un messaggio, 
compilando un array con la seguente struttura:

\begin{itemize}
	\item nell'indice 0 viene inserito il tipo di azione, se WRITE o READ.
	\item nell'indice 1 viene inserito l'indice del file descriptor della socket.
\end{itemize}
Una volta che il messaggio \`e stato ricevuto da RabbitMQ, lo inserisce all'interno della coda
gi\`a preconfigurata durante la fase di init.
Ad eseguire l'operazione di \textit{dequeue} \`e il consumer, che sfruttando un meccanismo
ad eventi, sveglia il thread dallo stato di \textit{WAIT} e lo porta in \textit{RUN}.
Una volta che RabbitMQ si accorge che tale messaggio \`e  stato preso in carico da un agente (nel nostro caso un'instanza del Consumer),
lo rimuove dalle coda, marcandolo come completato.
In questo modo il sistema \`e risultato fair e stabile durante la fase di test.

\begin{center}
\textbf{	Attenzione, il mapping tra connessione e thread non \`e strettamente 1:1}
\end{center}

Questo perch\`e quando il thread ha completato una richiesta, rilascia la socket attraverso il metodo \textit{producer\_release\_socket()}, riattivando dunque il producer a scansionare gli eventi relativi. Il thread che termina la richiesta, viene rimesso in attesa di messaggi
sulla coda di RabbitMQ.

\subsection{SQLite3}
Un'altra libreria che viene utilizzata \`e la famosa SQLite3, ovvero una libreria che permette di creare un database relazionale
all'interno di un singolo file con estensione .sqlite.
Questa scelta \`e stata fatta affinch\`e il progetto potesse avere una base solida, veloce e affidabile per la gestione dei dati.
In particolare, al posto di creare un sistema "artigianale" in caso per la memorizazione di strutture dati, sono state create
tabelle e query, per poter memorizzare utenti, messaggi e gruppi.

\paragraph{Tables}
Le tabelle che vengono create sono le seguenti:

\begin{figure}[H]
	\begin{table}[H]
		\begin{tabular}{lll}			
			\textbf{table} & \ \textbf{Primary key} & \textbf{description} \\
			users & nickname & contiene tutti gli utenti che si sono registrati \\
			messages & ID & contiene tutti i messaggi che sono stati inviati dagli utenti
		\end{tabular}
	\end{table}
\end{figure}

La gestione del database \`e completamente affidata al worker, ovvero a quel file descritto nella documentazione
del codice che contiene tuttte le funzionalit\'a per la persistenza dei dati.
Come \textbf{P e C} anche il worker deve essere inizializzato e distrutto ma non implementa un modello asincrono.
Le funzioni del worker sono tutte thread safe, dunque non c'\`e bisogno di implementare meccanismi di sincronismo, a
differenza del datastore.

\subsection{Libconfig}
Questa libreria viene utilizzata dalla maggior parte del mondo opensource per la lettura e il parsing di file
di configurazione, infatti \`e gi\`a installata nelle distro come Ubuntu, Debian e Centos.
La sua inizializzazione viene eseguita subito dopo il controllo degli argomenti passati al server.
Nel caso in cui il file di configurazione non esista, sia illeggibile (parlando di permessi), o contenga
un errore di sintassi, viene lanciato immediatamente un errore (\textit{config error}).

\paragraph{Formato del file di configurazione}
Il formato del file di configurazione, come specificato nella documentazione della libreria, \`e quello standard,
ovvero del tipo:
\begin{lstlisting}
	<conf-name> = <conf-value>
	<conf-value> := "<string-value>" | <other-types>
\end{lstlisting}
Gli altri tipi di configurazione messi a disposizione da LibConfig, come per esempio quelli per gli oggetti,
non sono stati utilizzati per questo progetto.
C'\`e da fare particolare attenzione alla differenza ai file \textit{.conf1} e \textit{.conf2}, perch\`e sono stati
adeguati al formato richiesto dalla libreria, cio\`e sono stati aggiunti i doppi apici all'inizio e alla fine di un
parametro stringa.

\subsection{Log}
La libreria di log \`e quella pi\`u semplice che \`e stata utilizzata. Non richiede l'installazione di file .so,
perch\`e \`e inclusa nel progetto stesso. Infatti il file \textit{log.c} e il relativo header \textit{log.h} contengono
le funzioni per la gestione dei logs.
Principalmente ci sono 4 livelli di log che possono essere utilizzati, e sono:

\begin{itemize}
	\item \textit{FATAL}
	\item \textit{ERROR}
	\item \textit{WARN}
	\item \textit{INFO}
	\item \textit{DEBUG}
	\item \textit{TRACE}
\end{itemize}

\`E anche possibile usufruire di un ulteriore funzionalit\`a chiamata \textit{QUITE}, che consente
di sopprimere tutti i tipi di errori che sono stati lanciati.
Questa libreria \`e anche stata resa dal sottoscritto thread safe, utilizzando un meccanismo
di mutua esclusione.

\section{Struttura del codice}
Il codice consegnato in realt\`a non \`e stato sviluppato nella struttura cos\`i  come si trova,
bens\`i sono state generate le directory

\begin{itemize}
	\item \textit{src} per tutti i file \textit{.c}
	\item \textit{include} per tutti i file \textit{.h}
	\item \textit{scripts} per tutti i file \textit{.sh}
	\item \textit{makefiles} per tutti i Makefile che sono stati testati
	\item \textit{DebugGCC/DebugClang/Release} per tutti i file processati dai rispettivi compilatori
\end{itemize}

Per rendere la struttura del progetto conforme a quella richiesta, \`e stato creato uno script
build\_project che esegue la copia dei file dentro le directory nella directory \textit{Release}.
Infatti i file consegnati sono esattamente quelli dentro la directory Release.
Se ci fosse necessit\`a di ispezionare gli ulteriori file del progetto \`e possibile scaricare
l'intero progetto dalla repository git da:
\url{https://git.com.divisible.net/federicosilvestri/chatty}
che per\`o richiede l'autenticazione da parte dello studente.

\section{Configurazione del server}
Oltre ai parametri richiesti nelle specifiche sono stati aggiunti altri parametri, descritti
nella seguente tabella:

\begin{figure}[H]
	\begin{table}[H]
		\begin{tabular}{lll}			
			\textbf{name} & \textbf{type} & \textbf{description} \\
			$RabbitMQHostname$ & string & l'hostname del server RabbitMQ \\ 
			$RabbitMQPort$ & integer & la porta sulla quale \`e in ascolto il server RabbitMQ \\
			$RabbitExchange$ & string & il nome dell'Exchange da creare per lo scambio dei messaggi \\
			$RabbitBindKey$ & string & chiave per proteggere l'exchange da binding sconosciuti. \\
		\end{tabular}
	\end{table}
\end{figure}

\section{Piattaforme testate}
Il progetto \`e stato sviluppato usando un sistema di Continuos Testing, fornito da \href{https://www.gitlab.com/}{Gitlab}.
Pertanto ogni volta che veniva eseguito un commit nel master branch,  seguendo il file di configurazione
\textit{.gitlab-ci.yml} (non fornito nella consegna), il sistema buildava il progetto ed eseguiva i test sotto ambienti:
\begin{itemize}
	\item Ubuntu 18.04
	\item Ubuntu 16.04
	\item Debian 9
\end{itemize}
In caso di test negativo, \`e stato ribuildato il progetto sulla macchina locale, in modo da debuggare al meglio
i problemi incontrati.

\subsection{Problemi che possono verificarsi}
Il problema che pu\`o verificarsi, durante la fase di testing \`e un errore che poteva essere soppresso,
ed \`e dovuto al problema dell'incompletezza del protocollo usato dal client.
In sostanza si verifica perch\`e non c'\`e modo di capire il momento in cui il client \`e in ascolto per nuovi
messaggi oppure \`e in ascolto per altri motivi.
Infatti nel protocollo seguito dal client per qualsiasi cosa si voglia fare viene inviata una richiesta
al server con il codice dell'operazione, tranne per la ricezione di messaggi.
In questo modo il server non \`e in grado di capire le intenzioni del client, e dunque utilizza una
strategia brutale ma funzionante, cio\`e utilizza il terzo parametro della funzione $select$ per capire
quando c'\`e spazio disponibile sul buffer della socket e se l'utente \`e autenticato, allora
viene inserito nella coda un messaggio di tipo $WRITE$. Alcune volte si verifica il messagio di 
errore "Broken pipe", ed \`e riferito al fatto che il server tentando di controllare lo stato dell'utente
(ormai disconnesso), riceve dal kernel la disconnessione.
Tuttavia questo errore che non \`e stato possibile evitare, se non:
\begin{itemize}
	\item modificando il protocollo del client
	\item aggiungendo una sleep durante la fase di ricezione
\end{itemize}
non ha prodotto problemi rilevanti durante il runtime del client, dunque anche durante i test.

\pagebreak

\section{Riferimenti}
\begin{figure}[H]
	\begin{table}[H]
		\begin{tabular}{ll}
			Gitlab & \url{www.gitlab.com} \\
			RabbitMQ & \url{www.rabbitmq.com} \\
			SQLite3 & \url{www.sqlite3.org} \\
			LibConfig & \url{https://hyperrealm.github.io/libconfig} \\
		\end{tabular}
	\end{table}
\end{figure}



\end{document}
